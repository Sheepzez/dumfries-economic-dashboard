% This example An LaTeX document showing how to use the l3proj class to
% write your report. Use pdflatex and bibtex to process the file, creating 
% a PDF file as output (there is no need to use dvips when using pdflatex).

% Modified 

\documentclass{l3proj}

\begin{document}

\title{Team Project(H) Dissertation: Team A}

\author{Lewis Dicks - 2085749 \\
        Isaac Jordan - 2080466 \\
        Praxitelis Kourtellos - 2060408 \\
        Christos Nicolaides - 2084564  \\
        Rostislav Yordanov - 2074214 \\
        Michael Byars - 2028262}

\date{\today}

\maketitle

\begin{abstract}

For the first time at the University of Glasgow, the team project undertaken in the third year of a Computing Science
degree involves a real life customer, where previously students undertook projects for individuals lecturers in the
department. This was also the first year in which the Team Project was combined with the Professional Software
Development course as previously students were involved in two team projects, this is now one team with one project that
incorporates techniques learned from the Professional Software Development course. Students worked with institutions
based in the Dumfries and Galloway area of Scotland, such as the Crichton Regional Observatory, the NHS and the third
sector. For our team's project, we were asked to create an interactive dashboard of various economic indicators that
would be integrated into the Crichton's homepage. This paper presents a case study of the nature of our project and
reflects on our experiences relating to software engineering practices and principles.

\end{abstract}

%% Comment out this line if you do not wish to give consent for your
%% work to be distributed in electronic format.
\educationalconsent

\newpage

%==============================================================================
\section{Introduction}

For our team's project, we were asked to create an interactive dashboard of various economic indicators that would be
integrated into the Crichton Institutes's homepage. The idea behind this project is to provide a easy to access/instant
visualisation of key economic data that allows businesses, agencies etc, to gain a snapshot of the current state of the
regional economy without having to trawl through lots of data links. Our project was built using the Python-based Django
framework as we have had previous universty experience building web applicaions using this so it was relatively
straightforward to start developing the initial components of the back-end of the website. In addition to this we used
many JavaScript-based front end technolgies such as Angular, d3 and Gridster. 

% TODO: talk more about the basis of our project i.e. technolgies used, rough outline of the content in the site


%% Final paragraph.
The rest of the case study is structured as follows.  Section
\ref{sec:background} presents the background of the case study
discussed, describing the customer and project context, aims and
objectives and project state at the time of writing.  Sections
\ref{sec:alice} through Section \ref{sec:managing} discuss issues that
arose during the project...

%==============================================================================
\section{Case Study Background}

Include details of 

\begin{itemize}
\item The customer organisation and background.
\item The rationale and initial objectives for the project.
\item The final software was delivered for the customer.
\end{itemize}

% This is the background I obtained from Trac - Lewis

Crichton Institute (CI) was launched in January 2013. It is a unique collaborative venture between the Crichton Campus academic
institutions and wider partners in the business, local government, health and voluntary sectors.

CI is a focal point for all aspects of research and development relating to south-west Scotland with the view to understanding the
issues that constrain the region, to develop an evidence base for its economic, social and cultural development and to raise its
national and international profile. Crichton Institute works through a series of inter-related activities: Research;
Business Intelligence, Policy development; and, a Regional Observatory (RO).

RO is a web-based information and knowledge portal that acts as a one-stop open access service for public data, information and
intelligence about a wide range of social, economic and environmental factors across Dumfries and Galloway and the South of Scotland.

So far, it appears that RO is the only exclusively rural data observatory. All other projects they have come across in the UK and abroad
are based predominantly on urban/city regions. It is vital that RO provides information that allows individuals, businesses and potential
investors to gain access to important datasets and reports that allow them to make decisions that benefit the region.

Our customers representing the Crichton Institute were Development Officer Eva Milroy and Director Tony Fitzgerald, who were the key
focal points in terms of our team demonstrating progress at th end of each iteration and reporting feedback from their shareholders.

% Their problem description

"We would like RO to provide quick, virtually instant access to headline information on the current key economic indicators that are of
importance to Dumfries and Galloway. Information on current regional unemployment/employment figures, GVA, house prices, business start-ups,
population etc, should be available via the home page of RO, as well as the CI website, with links to the relevant sources
(i.e. NOMIS, Scot Stats etc). The exact details of data that can most usefully be included is open for discussion. Ideally, we would like
the economic dashboard to look similar to those of the London Dashboard or that of the Dublin Dashboard.

Ideally we would like APIs that link to the relevant local data, offer a comparison to the Scotland-wide data and,
if possible even a UK-wide comparison.

If the project time allows, we would like the functionality to be built in that the user is able to ‘build’ his/her own cheat sheet,
a summary of the indicators important to him/her, and download this as a printable pdf document."

%TODO: talk about what we initally agreed with Eva, and what we eventually delivered at the end of the project

%==============================================================================
\section{Alice}

%TODO: not 100% sure exactly what we need to talk about but we should look back on the retrospectives to help out


%==============================================================================

\section{Choice of Colours}
\label{design}

The following diagrams illustrate the
process...

%==============================================================================
\section{Managing Dress Sense}
\label{managing}

In this chapter, we describe how the implemented the system.

%------------------------------------------------------------------------------
\section{Kangaroo Practices}



% - - - - - - - - - - - - - - - - - - - - - - - - - - - - - - - - - - - - - - -
\section{Knots and Bundles}
\label{sec:managing}


%------------------------------------------------------------------------------
\section{Conclusions}

Explain the wider lessons that you learned about software engineering,
based on the specific issues discussed in previous sections.  Reflect
on the extent to which these lessons could be generalised to other
types of software project.  Relate the wider lessons to others
reported in case studies in the software engineering literature.

%==============================================================================
\bibliographystyle{plain}
\bibliography{dissertation}
\end{document}
