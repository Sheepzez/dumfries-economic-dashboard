% This example An LaTeX document showing how to use the l3proj class to
% write your report. Use pdflatex and bibtex to process the file, creating 
% a PDF file as output (there is no need to use dvips when using pdflatex).

% Modified 

\documentclass{l3proj}

\begin{document}

\title{Team Project(H) Dissertation: Team A}

\author{Lewis Dicks - 2085749 \\
        Isaac Jordan - 2080466 \\
        Praxitelis (Branko) Kourtellos - 2060408 \\
        Christos (Takis) Nicolaides - 2084564  \\
        Rostislav Yordanov - 2074214 \\
        Michael Byars - 2028262}

\date{\today}

\maketitle

\begin{abstract}

For the first time at the University of Glasgow, the team project undertaken in the third year of a Computing Science
degree involves a real life customer, where previously students undertook projects for individuals lecturers in the
department. This was also the first year in which the Team Project was combined with the Professional Software
Development course as previously students were involved in two team projects, this is now one team with one project that
incorporates techniques learned from the Professional Software Development course. Students worked with institutions
based in the Dumfries and Galloway area of Scotland, such as the Crichton Regional Observatory, the NHS and the third
sector. For our team's project, we were asked to create an interactive dashboard of various economic indicators that
would be integrated into the Crichton's homepage. This paper presents a case study of the nature of our project and
reflects on our experiences relating to software engineering practices and principles.

\end{abstract}

%% Comment out this line if you do not wish to give consent for your
%% work to be distributed in electronic format.
\educationalconsent

\newpage

%==============================================================================
\section{Introduction}

This project report covers a 7 month University of Glasgow third year project. For 
our team's project, we were asked to create an interactive dashboard of various 
economic indicators that would be integrated into the Crichton Institutes's homepage. The idea behind this project is to provide an easy to access/instant
visualisation of key economic data that allows businesses, agencies etc, to gain a snapshot of the current state of the
regional economy without having to trawl through lots of data links. Our project was built using the Python-based Django
framework as the team had previous university experience building web applications using this so it was relatively
straightforward to begin developing the initial components of the back-end of the website. In addition to this we used
many JavaScript-based front end technologies such as Angular JS~\cite{AngularWebpage}, d3~\cite{d3Webpage} and Gridster.




%% Final paragraph.
The rest of the case study is structured as follows.  Section
\ref{sec:background} presents the background of the case study
discussed, describing the customer and project context, aims and
objectives and project state at the time of writing.  Sections
Bleh through Section Blah discuss issues that
arose during the project...

%==============================================================================
\section{Case Study Background} \label{sec:background}

Include details of 

\begin{itemize}
\item The customer organisation and background.
\item The rationale and initial objectives for the project.
\item The final software was delivered for the customer.
\end{itemize}

% This is the background I obtained from Trac - Lewis

Crichton Institute (CI)~\cite{CrichtonInstitute} was launched in January 2013. It is a unique collaborative venture between the Crichton Campus academic
institutions and wider partners in the business, local government, health and voluntary sectors.

CI is a focal point for all aspects of research and development relating to south-west Scotland with the view to understanding the
issues that constrain the region, to develop an evidence base for its economic, social and cultural development and to raise its
national and international profile. Crichton Institute works through a series of inter-related activities: Research,
Business Intelligence, Policy development, and a Regional Observatory (RO).

RO is a web-based information and knowledge portal that acts as a one-stop open-access service for public data, information and
intelligence about a wide range of social, economic and environmental factors across Dumfries and Galloway and the South of Scotland.

Our customers representing the Crichton Institute were Development Officer Eva Milroy and Director Tony Fitzpatrick, who were the key
focal points in terms of our team demonstrating progress at the end of each iteration and reporting feedback from their shareholders.

% Their problem description

"We were asked to provide quick, virtually instant access to headline information on the current key economic indicators that are of
importance to Dumfries and Galloway. Information on current regional unemployment/employment figures, GVA, house prices, business start-ups,
population etc, should be available via the home page of RO, as well as the CI website, with links to the relevant sources.
The exact details of data that can most usefully be included was open for discussion.

They were asking about the possibility of using APIs that link to the relevant local data, offer a comparison to the Scotland-wide data and,
if possible even a UK-wide comparison. If the project time allowed, they wanted functionality to be built in that the user is able to ‘build’ his/her
own cheat sheet,a summary of the indicators important to him/her, and download this as a printable pdf document."


Ideally, they wanted the dashboard to look similar to those of the London Dashboard~\cite{LondonDashboard} or that of the Dublin Dashboard\cite{DublinDashboard}.

%TODO: talk about what we initally agreed with Eva, and what we eventually delivered at the end of the project

%==============================================================================
\section{Team Dynamics}

We talk about initial individual efforts, how more team members contributed more as the site grew into something functional
and the importance tof documentation etc, our use of slack for team communication with integration with SVN/Jenkins

The assignment of members to project teams was randomised and sorted based on degree discipline, with our team being
Computing Science oriented. Our team was randomised, however two of us were close friends and another two were from the
same country which helped with initial team cohesion. One of our first tasks was to add one another on Facebook as a means of
team communication, however in addition to this we also used Slack, a cloud-based team collaboration tool. We found this to
be of much better use than Facebook due to the organisation of discussion groups (e.g. one for general communication, one for
minutes of meetings). We were also able to integrate Jenkins and SVN into this tool to monitor the success of project builds,
and also to monitor when new commits are made to the repository. We had Jenkins integration early on in the project life cycle
but introduced the SVN component approximately half way through the process, in hindsight we wished we had used this from the
start. At the early stages we also organised team roles: Lewis as Project Manager; Isaac as Tech Lead; Michael as Client Liason;
Ross as Retrospective Manager and Branko and Takis were Software Engineers. In terms of agreement of these roles they were unanimous
based on the persoanl traits of each team member and this was consistently adhered to throughout the project.

One particular observation we noticed early on in the process was the commitment of team members to team meetings and also
to efforts made in developing the project. The university had timetabled one all day lab session per week that was dedicated
to team project work, and on these days we had the majority of our team members attend. The problems seemed to arise when
team members did not show up for additional team meetings arranged well in advance, sometimes this was due to mittigating
circumstances such as ill health or part-time job commitments, and whilst this was acceptable the main issue was the fact
that at this stage team members seemed to give no prior notice that they would not be available and consequently only
half the team showed up to these meetings on average. As the process continued the consistency of team members attending these
extra meetings improved albeit if they showed up later than intially arranged.

In terms of effort put into the project, there seemed to be a pattern where it was only our Tech Lead that was committing the most
work into the project in the initial stages. This was down to the fact he has professional web development experience from previous
years. However once the site was partially functioning as a proof of concept, it encouraged more team members to make contributions
to the project.


%==============================================================================

\section{System Implementation}
\label{design}

We should talk about the building stages of the site, making the build component based, availability of APIs

%==============================================================================
\section{Process Improvements}
\label{managing}

We talk about the improvements we've made as a team, such as improvements make to ticket management

%------------------------------------------------------------------------------

\section{Customer Management}
\label{sec:managing}

We talk about how we appointed a customer liason to communicate with and receive feedback from our customer, how we dealt with
demonstrations, how we dealt with change of expectations and changes to customer requirements. We also should talk about the
introduction of requirements from other teams in terms of integration into the Crichton site.


%------------------------------------------------------------------------------
\section{Conclusions}

Explain the wider lessons that you learned about software engineering,
based on the specific issues discussed in previous sections.  Reflect
on the extent to which these lessons could be generalised to other
types of software project.  Relate the wider lessons to others
reported in case studies in the software engineering literature.

%==============================================================================
\bibliographystyle{plain}
\bibliography{dissertation}
\end{document}
