% This example An LaTeX document showing how to use the l3proj class to
% write your report. Use pdflatex and bibtex to process the file, creating 
% a PDF file as output (there is no need to use dvips when using pdflatex).

% Modified 

\documentclass{l3proj}

\begin{document}

\title{Team Project(H) Dissertation: Team A}

\author{Lewis Dicks - 2085749 \\
        Isaac Jordan - 2080466 \\
        Praxitelis Kourtellos - 2060408 \\
        Christos Nicolaides - 2084564  \\
        Rostislav Yordanov - 2074214 \\
        Michael Byars - 2028262}

\date{\today}

\maketitle

\begin{abstract}

For the first time at the University of Glasgow, the team project undertaken in the third year of a Computing Science
degree involves a real life customer, where previously students undertook projects for individuals lecturers in the
department. This was also the first year in which the Team Project was combined with the Professional Software
Development course as previously students were involved in two team projects, this is now one team with one project that
incorporates techniques learned from the Professional Software Development course. Students worked with institutions
based in the Dumfries and Galloway area of Scotland, such as the Crichton Regional Observatory, the NHS and the third
sector. For our team's project, we were asked to create an interactive dashboard of various economic indicators that
would be integrated into the Crichton's homepage. This paper presents a case study of the nature of our project and
reflects on our experiences relating to software engineering practices and principles.

\end{abstract}

%% Comment out this line if you do not wish to give consent for your
%% work to be distributed in electronic format.
\educationalconsent

\newpage

%==============================================================================
\section{Introduction}

Software engineering 

This paper presents a case study of... 


%% Final paragraph.
The rest of the case study is structured as follows.  Section
\ref{sec:background} presents the background of the case study
discussed, describing the customer and project context, aims and
objectives and project state at the time of writing.  Sections
\ref{sec:alice} through Section \ref{sec:managing} discuss issues that
arose during the project...

%==============================================================================
\section{Case Study Background}

Include details of 

\begin{itemize}
\item The customer organisation and background.
\item The rationale and initial objectives for the project.
\item The final software was delivered for the customer.
\end{itemize}

%==============================================================================
\section{Alice}


%==============================================================================

\section{Choice of Colours}
\label{design}

The following diagrams illustrate the
process...

%==============================================================================
\section{Managing Dress Sense}
\label{managing}

In this chapter, we describe how the implemented the system.

%------------------------------------------------------------------------------
\section{Kangaroo Practices}



% - - - - - - - - - - - - - - - - - - - - - - - - - - - - - - - - - - - - - - -
\section{Knots and Bundles}
\label{sec:managing}


%------------------------------------------------------------------------------
\section{Conclusions}

Explain the wider lessons that you learned about software engineering,
based on the specific issues discussed in previous sections.  Reflect
on the extent to which these lessons could be generalised to other
types of software project.  Relate the wider lessons to others
reported in case studies in the software engineering literature.

%==============================================================================
\bibliographystyle{plain}
\bibliography{dissertation}
\end{document}
