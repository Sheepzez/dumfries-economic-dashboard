% This example An LaTeX document showing how to use the l3proj class to
% write your report. Use pdflatex and bibtex to process the file, creating 
% a PDF file as output (there is no need to use dvips when using pdflatex).

% Modified 

\documentclass{l3proj}

\begin{document}

\title{Team Project(H) Dissertation: Team A}

\author{Lewis Dicks - 2085749 \\
        Isaac Jordan - 2080466 \\
        Praxitelis (Branko) Kourtellos - 2060408 \\
        Christos (Takis) Nicolaides - 2084564  \\
        Rostislav Yordanov - 2074214 \\
        Michael Byars - 2028262}

\date{\today}

\maketitle

\begin{abstract}

For the first time at the University of Glasgow, the team project undertaken in the third year of a Computing Science
degree involves a real life customer, where previously students undertook projects for individuals lecturers in the
department. This was also the first year in which the Team Project was combined with the Professional Software
Development course as previously students were involved in two team projects, this is now one team with one project that
incorporates techniques learned from the Professional Software Development course. Students worked with institutions
based in the Dumfries and Galloway area of Scotland, such as the Crichton Regional Observatory, the NHS and the third
sector. For our team's project, we were asked to create an interactive dashboard of various economic indicators that
would be integrated into the Crichton's homepage. This paper presents a case study of the nature of our project and
reflects on our experiences relating to software engineering practices and principles.

\end{abstract}

%% Comment out this line if you do not wish to give consent for your
%% work to be distributed in electronic format.
\educationalconsent

\newpage

%==============================================================================
\section{Introduction}

This project report covers a 7 month University of Glasgow third year project. For 
our team's project, we were asked to create an interactive dashboard of various 
economic indicators that would be integrated into the Crichton Institutes's homepage. The idea behind this project is to provide an easy to access/instant
visualisation of key economic data that allows businesses, agencies etc, to gain a snapshot of the current state of the
regional economy without having to trawl through lots of data links. Our project was built using the Python-based Django
framework as the team had previous university experience building web applications using this so it was relatively
straightforward to begin developing the initial components of the back-end of the website. In addition to this we used
many JavaScript-based front end technologies such as Angular JS~\cite{AngularWebpage}, d3~\cite{d3Webpage} and Gridster.




%% Final paragraph.
The rest of the case study is structured as follows.  Section
\ref{sec:background} presents the background of the case study
discussed, describing the customer and project context, aims and
objectives and project state at the time of writing.  Sections
Bleh through Section Blah discuss issues that
arose during the project...

%==============================================================================
\section{Case Study Background} \label{sec:background}

Include details of 

\begin{itemize}
\item The customer organisation and background.
\item The rationale and initial objectives for the project.
\item The final software was delivered for the customer.
\end{itemize}

% This is the background I obtained from Trac - Lewis

Crichton Institute (CI)~\cite{CrichtonInstitute} was launched in January 2013. It is a unique collaborative venture between the Crichton Campus academic
institutions and wider partners in the business, local government, health and voluntary sectors.

CI is a focal point for all aspects of research and development relating to south-west Scotland with the view to understanding the
issues that constrain the region, to develop an evidence base for its economic, social and cultural development and to raise its
national and international profile. Crichton Institute works through a series of inter-related activities: Research,
Business Intelligence, Policy development, and a Regional Observatory (RO).

RO is a web-based information and knowledge portal that acts as a one-stop open-access service for public data, information and
intelligence about a wide range of social, economic and environmental factors across Dumfries and Galloway and the South of Scotland.

Our customers representing the Crichton Institute were Development Officer Eva Milroy and Director Tony Fitzpatrick, who were the key
focal points in terms of our team demonstrating progress at the end of each iteration and reporting feedback from their shareholders.

% Their problem description

"We were asked to provide quick, virtually instant access to headline information on the current key economic indicators that are of
importance to Dumfries and Galloway. Information on current regional unemployment/employment figures, GVA, house prices, business start-ups,
population etc, should be available via the home page of RO, as well as the CI website, with links to the relevant sources.
The exact details of data that can most usefully be included was open for discussion.

They were asking about the possibility of using APIs that link to the relevant local data, offer a comparison to the Scotland-wide data and,
if possible even a UK-wide comparison. If the project time allowed, they wanted functionality to be built in that the user is able to ‘build’ his/her
own cheat sheet,a summary of the indicators important to him/her, and download this as a printable pdf document."

Ideally, they wanted the dashboard to look similar to those of the London Dashboard~\cite{LondonDashboard} or the Dublin Dashboard\cite{DublinDashboard}.

% What we initially agreed compared to what we delivered at the end

Of all the requirements we initially agreed upon with the customers, the majority of them have been addressed and either completed as initially suggested
or an alternative has been put in place. The only large initially discussed task we will not be completing is building our own API but it was concluded
with Eva and Tony during a meeting that this was not necessary anyway and because there were a lot of issues concerning availability of data, it would be
extremely difficult and we could use the time we would have spent working on that more effectively, improving other, more important parts of the site.

The customers have been consistently happy throughout, and not at any point did they ask us to remove something we had done, except for small details such 
as shadowing behind text. They made various suggestions during the meetings but even most of those were ideas that we had already thought about but were 
yet to implement.

Overall the final product fits the initial brief almost exactly and the customer is happy that what we have produced meets their requirements. That
is, afterall, the most important thing.


% (Under Construction)

%==============================================================================
\section{Team Dynamics}

We talk about initial individual efforts, how more team members contributed more as the site grew into something functional
and the importance tof documentation etc, our use of Slack for team communication with integration with SVN/Jenkins

The assignment of members to project teams was randomised and sorted based on degree discipline, with our team being
Computing Science oriented. Our team was randomised, however two of us were close friends and another two were from the
same country which helped with initial team cohesion. One of our first tasks was to add one another on Facebook as a means of
team communication, however in addition to this we also used Slack, a cloud-based team collaboration tool. We found this to
be of much better use than Facebook due to the organisation of discussion groups (e.g. one for general communication, one for
minutes of meetings). We werearly on in the project life cycle
but introduced the SVN component approximately half way through the process, e also able to integrate Jenkins and SVN into this tool to monitor the success of project builds,
and also to monitor when new commits are made to the repository. We had Jenkins integration in hindsight we wished we had used this from the
start. At the early stages we also organised team roles: Lewis as Project Manager; Isaac as Tech Lead; Michael as Client Liason;
Ross as Retrospective Manager and Branko and Takis were Software Engineers. In terms of agreement of these roles they were unanimous
based on the persoanl traits of each team member and this was consistently adhered to throughout the project.

One particular observation we noticed early on in the process was the commitment of team members to team meetings and also
to efforts made in developing the project. The university had timetabled one all day lab session per week that was dedicated
to team project work, and on these days we had the majority of our team members attend. The problems seemed to arise when
team members did not show up for additional team meetings arranged well in advance, sometimes this was due to mittigating
circumstances such as ill health or part-time job commitments, and whilst this was acceptable the main issue was the fact
that at this stage team members seemed to give no prior notice that they would not be available and consequently only
half the team showed up to these meetings on average. As the process continued the consistency of team members attending these
extra meetings improved albeit if they showed up later than intially arranged.

In terms of effort put into the project, there seemed to be a pattern where it was only our Tech Lead that was committing the most
work into the project in the initial stages. This was down to the fact that he has professional web development experience from previous
years while the rest of the team have limited experience. However once the site was partially functioning as a proof of concept, it 
encouraged more team members to make contributions to the project.


%==============================================================================

\section{System Implementation}
\label{design}

We should talk about the building stages of the site, making the build component based, availability of APIs

In software engineering there are many ways to implement a project but there are certain tools that are more advatageous over
others. For our team project we were asked to build a web application, so in doing so we had to deicde on a web framework that
we could use as the basis of our project. We chose the Python-based Django framework as we have all have had previous experience
in developing web applications using this as part of a unviersity course. There were options where we could have used an alternative
framework such as Java Spring, and whilst we are all proficient as team members in the use of Java, there would be too much time
wasted on learning the basics of Spring. In choosing Django it allowed us to spend more time on the project itself. Our project
used a variety of languages such as HTML, CSS, Bootstrap and Javascript, all in which we have had previous experience using.
We also used new technologies such as Angular JS, d3 for graph visualisation and Grdister for the movement and resizing of graph
elements.

The starting point of our project was to create a base template for the site: we found a Boostrap template that was a good starting
point in terms of a colour scheme and layout that was both simplistic yet professional. Next was to decide how to implement the
graphs: we decided upon using d3 for graph visualisation as this was mentioned by a lecturer in another course we were currently
taking at university. To begin with the graphs were taken from a demonstration website with hard-coded data as more of a proof
of concept that we could implement bar and line graphs. After this we wanted a way of making the dashboard customisable, so we
found Grdister to be very effective in terms of moving and resizing graphs. By the deadline of the first iteration this was
achieved and was a great starting point to develop our site upon.

The next issue we had to deal with is how we can use real data in these graphs. We started by researching the various websites
suggested by our customer such as NOMIS Scotland. We found that most of these sites were difficult to use in terms of extracting
exactly the kind of data that we needed. Eventually we came across Statistics beta which was a site being set up on behalf of
the Scottish Government to be used as a centralised access point to informatiob. In this, we were able to find data on
hundreds of economic indicators in CSV format. We created a script that would process these files, extract only the data collected
on Dumfries and Galloway and Scotland and format this to be rendered by the d3 code that visualises the data before storing it on
our site's database. This was a major step in our project in terms of using real data, however our main issue here was that the
data was static, and at the time of download was outdated. In addition to this the files all used inconsistent date formats (e.g.
one would be quarterly, another would be monthly, another yearly) which presented a major issue in terms of a key requirement
being that users want to access the latest available data as well as looking back on previous years. We raised this issue
with our customers and they were aware of this issue beforehand, this was a relief.

%==============================================================================
/section{Core Practices}
/label {Agile Software Development}

Throughout the development of the project we were also following a course named Professional Software Development 3.
The course's aim was to introduce the students to modern software development methods and techniques for building and
maintaining large systems and apply those techniques in a project (which is this ) . One of the methods that were taught
was the the agile software development. Agile software development "is a set of principles for software development in
which requirements and solutions evolve through collaboration between self-organizing, cross-functional teams.
It promotes adaptive planning, evolutionary development, early delivery, and continuous improvement, and it encourages
rapid and flexible response to change".

The agile method that was mostly used was Extreme Programming(XP). The benefits of XP is beneficial since it focuses on
customer satisfaction, enables the continuous elicitation of requirements from the customers, it insists on teamwork,
both between the team members and the customers. Also, XP improves the communication, simplicity, feedback, respect and
courage of the team.Last XP focuses on a number of simple  rules that can be easily remembered.

Following the guidelines of XP we concluded in the following scheme.Since the project  was set to finish in a specific date
we had been set a number of dates as the end of the iteration.In total 5 iterations(including the final demonstration)
have been set from the university. The first iteration was  our initial metting with the customers.
From this point we have understood the concept of the project, collected  the initial requirements of the project
and also proposed some ideas to our clients. After the end of the meeting the requirements have been translated into user
 stories, from where a number of researches have been initialised to provide possible solution to the user stories. By the
 end of the iteration a meeting with the customers has been reinvoked demonstarting the current structure of the project,
 asking whether it fits their needs and collect new requirements for the following iterations. This scheme has been followed
 until the end of the project and thus the final demonstration. It is also fair to say that inbetween the iterations we
 sought defects which got translated into new user stories and fixed during the iteration.

//PLEASE SOMEONE LOOK UP TO WHAT I'M WRITTING !<3

%==============================================================================
\section{Process Improvements}
\label{managing}

We talk about the improvements we've made as a team, such as improvements make to ticket management

%------------------------------------------------------------------------------

\section{Customer Management}
\label{sec:managing}

We talk about how we appointed a customer liason to communicate with and receive feedback from our customer, how we dealt with
demonstrations, how we dealt with change of expectations and changes to customer requirements. We also should talk about the
introduction of requirements from other teams in terms of integration into the Crichton site.


%------------------------------------------------------------------------------
\section{Conclusions}

Explain the wider lessons that you learned about software engineering,
based on the specific issues discussed in previous sections.  Reflect
on the extent to which these lessons could be generalised to other
types of software project.  Relate the wider lessons to others
reported in case studies in the software engineering literature.

%==============================================================================
\bibliographystyle{plain}
\bibliography{dissertation}
\end{document}
